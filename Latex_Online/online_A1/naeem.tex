\documentclass[12pt,a4paper]{article}
\usepackage[margin=1in]{geometry}
\usepackage[utf8]{inputenc}
\usepackage{color}
\usepackage{enumerate}
\usepackage{multirow}
\usepackage{multicol}
\usepackage{graphicx}
\usepackage{amsmath}
\usepackage{calligra}
\usepackage{hyperref}

\title{CSE 300: Online Assignment}
\author{
	Md Shamsuzzoha Bayzid$^1,*$, Mahjabin Nahar$^1,+$, Md Shariful Islam\\ Bhuyan$^1,+$, and Md Saidur Rahman$^1,+$ \\ \\
	$^1$Department of Computer Science and Engineering \\
	Bangladesh University of Engineering and Technology \\
	*Corresponding author: shams bayzid@cse.buet.ac.bd \\
	These authors contributed equally to this work
}

\date{April 07, 2021}

\begin{document}
	\maketitle

	\section{Introduction}
	This assignment has been designed to assess the preparation of the students in writing
	scientific articles using \LaTeX\ . Different components, that are frequently used in scientific manuscripts, have been covered in this assignment.
	
	\subsection{Tables}
	We wish to place Table~\ref{Table:1} right here. \newline
	\begin{table}[h]
		\centering
		\caption{\textbf{Optimization scores for Method-1 and Method-2 on different datasets covering various model conditions.} We show average scores of two optimization criteria for various model conditions.\newline} 
		\label{Table:1}
		
		\begin{tabular}{| c | c c | c c | c c |}
			
			\hline 
			\multicolumn{3}{|c|}{Simulation Condition} & \multicolumn{4}{|c|}{Optimization Score} \\
			\hline
			\multirow{2}{*}{Dataset}  & \multirow{2}{*}{Complexity}  & Model & \multicolumn{2}{|c|}{Score 1} & \multicolumn{2}{|c|}{Score 2}  \\
			\cline{4-7}
			& & condition & Method-1 & Method-2 & Method-1 & Method-2 \\
			\hline \hline
			
			\multirow{4}{*}{D1}  & \multirow{2}{*}{Easy}  & M1 & 7,425.55 & 770.00 & 929.55 & 10 \\
			 & & M2 & 7,657.00 & 9,179.00 & 716.15 & 20 \\
			 \cline{2-7}
			 & \multirow{2}{*}{Hard}  & M3 & 54.00 & 9,007.15 & 3,759.00 & 30 \\
			 & & M4 & 74.00 & 5567.15 & 99.00 & 25 \\
			 \hline \hline
			 \multirow{3}{*}{D3} & \multirow{3}{*}{Moderate}  & M1 & 34.00 & 273.00 & 321.60 & 34 \\
			 & & M2 & \multicolumn{2}{|c|}{ Not Applicable} & 16.00 & 11 \\
			 & & M3 & 657.00 & 179.60 & 716.00 & 19 \\
			 \hline
		\end{tabular}
	\end{table}
	\pagebreak
	
	\subsection{Figures}
	We intend to put Figure~\ref{figNNI} at the top of a page.
	
	\begin{figure}[t]
		\centering	
		\includegraphics[scale=0.32]{Figure3.pdf}
		\caption{\textbf{Nearest Neighbor Interchange (NNI) move on an internal edge.} (a)	A species tree \textit{ST}, and (b)-(c) the neighbors of \textit{ST} resulting from one NNI move on edge $e = (u1; u2)$. $A,B,C,$ and $D$ are the sets of taxa in the four subtrees around edge $e$.}
		\label{figNNI}
		
	\end{figure}
	
	
	
	\subsection{Mathematical Equations}
	Let $n1|n2|n3$ be a tripartition defined on an internal node $u$ of a binary tree $T$. The number of tripartitions mapped to $u$ is given by Eqn. ~\ref{eq1} .
	
	\begin{equation}
		\begin{aligned}
		\mathcal{NQ}(n1,n2,n3) = \binom{n_1}{2}\binom{n_2}{1}\binom{n_3}{1}+ \binom{n_2}{2}\binom{n_1}{1}\binom{n_3}{1} + \binom{n_3}{2}\binom{n_1}{1}\binom{n_2}{1} \\
		= \frac{n1n2n3(n1 + n2 + n3 + 3)}{2}
		\end{aligned}
	\end{equation}
	\section{Conclusions}
	The major objectives of this assignment are listed below (please do not ignore the font
	sizes).
	\begin{itemize}
		\item{ \Large To see if the students have adequately practiced different aspects of writing in \LaTeX\ .} 
		{\large \item  To assess the ability of the students in preparing manuscripts in \LaTeX\ .}
		\item To see if the students can add various basic components (e.g., tables, figures, equations) to a \LaTeX\ manuscript.
		{\small \item To see if the students can leverage the available materials (both offline and online) to do something which has not explicitly been taught in the class.}
		
	\end{itemize}


	
\end{document}